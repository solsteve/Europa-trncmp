%/ ====================================================================== BEGIN FILE =====
%/ **                                   A N O M A L Y                                   **
%/ =======================================================================================
%/ **                                                                                   **
%/ **  Copyright (c) 2019, Stephen W. Soliday                                           **
%/ **                      stephen.soliday@trncmp.org                                   **
%/ **                      http://research.trncmp.org                                   **
%/ **                                                                                   **
%/ **  -------------------------------------------------------------------------------  **
%/ **                                                                                   **
%/ **  This program is free software: you can redistribute it and/or modify it under    **
%/ **  the terms of the GNU General Public License as published by the Free Software    **
%/ **  Foundation, either version 3 of the License, or (at your option)                 **
%/ **  any later version.                                                               **
%/ **                                                                                   **
%/ **  This program is distributed in the hope that it will be useful, but WITHOUT      **
%/ **  ANY WARRANTY; without even the implied warranty of MERCHANTABILITY or FITNESS    **
%/ **  FOR A PARTICULAR PURPOSE. See the GNU General Public License for more details.   **
%/ **                                                                                   **
%/ **  You should have received a copy of the GNU General Public License along with     **
%/ **  this program. If not, see <http://www.gnu.org/licenses/>.                        **
%/ **                                                                                   **
%/ ----- Modification History ------------------------------------------------------------
%/
%/  @file Anomaly.tex
%/   Provides a description of a method for anomaly detection in high dimensionality data.
%/
%/  @author Stephen W. Soliday
%/  @date 2019-Jul-21 (final)
%/  @date 2019-Jun-25 (draft)
%/
%/ =======================================================================================

\documentclass{article}

\usepackage{amsmath}
\usepackage{amsfonts}
\usepackage{amssymb}
\usepackage{graphicx}
\usepackage{wrapfig}
\usepackage{trncmp}
\usepackage{picinpar}
\usepackage{asiwp}
\usepackage{environ}
\usepackage{fancyvrb}

%  =======================================================================================
% Heading arguments are {volume}{year}{pages}{submitted}{published}{author-full-names}

\jmlrheading{Program Whitepaper}{2019}{}{6/25}{-}{Stephen W.~Soliday}{L3T/ADS}

% Short headings should be running head and authors last names

\ShortHeadings{Anomaly Detection}{Soliday}
\firstpageno{1}
%  =======================================================================================


\begin{document}
\title{Auto-encoding Deep Neural Networks for Anomaly Detection}
\author{ \name Stephen W.~Soliday \email stephen.soliday@sg.l3t.com \\
  \addr L3 Technologies \\ Advanced Defense Solutions
}
\date{25-June-2019}

\maketitle

\begin{abstract}%

  Analysts have to sift through thousands of images looking for the one image that may be different.
  This approach uses unsupervised learning to discover relevant features in a mass of similar images.
  The system then sorts the images from least likly to most likelyto represent the archtype image.
  Two experiments are performed. One with image data, the second with highdimentional non-image data.
  
\end{abstract}

\vspace{12pt}
\textbf{Keywords}: Auto-encoding, Neural network, Unsupervised

%/ =======================================================================================
\section{Introduction\label{sec:intro}}
%/ =======================================================================================

A typical task for image analysts is to sift through thousands of similar images, looking
for that one image that is different. Deep Neural Networks have been very successfull at
learningto classifiy images, when presended with labeled data.

Imagine that you have a thousand small round stones in you driveway.
Amongst those thousand stones is one snail, and you must find the snail.
Prior to the task you have no knowleged about either stones or snails.

You would pick up and examine each object. Over time you would come to
understand what attributes or features are common to the mass and what is different.

Within the mass some objects would be the most like the average of the mass and some
objects that are very different from the mass.

Little if anything that you learn in this task may be used in the next task. Maybe next time
your task is to find the one dime in ajar of nickles.


%/ ---------------------------------------------------------------------------------------
\subsection{Background and Opposing Arguments\label{sec:back}}
%/ ---------------------------------------------------------------------------------------

This is a short, intermediate section that may be useful to include
between the Introduction and the Body of the paper, depending on the
topic. In this section you can briefly present some background
information that readers may need in order to understand and evaluate
the thesis support. 

In addition (or instead), this section may include a brief discussion of views or arguments 
that run counter to the paper’s thesis, yet have some merit and are relevant to the central issue.

\newpage

%/ =======================================================================================
\section{Auto Encoding Decoding Neural Network\label{sec:autoencoder}}
%/ =======================================================================================

\begin{figure}
  \begin{center}
    \includegraphics[width=4in]{figures/AutoGen.eps}
    \caption{Three layer feed forward neural network}
    \label{fig:threelayer}
  \end{center}
\end{figure}

Figure\ref{fig:threelayer}, shows the schematic for an auto-encoding/decoding neural network.


\newpage

%/ =======================================================================================
\section{Results\label{sec:results}}
%/ =======================================================================================

How you present the results of your research depends on what kind of
research you did, your subject matter, and your readers’
expectations. 

The distinction between the results section and the discussion section
is not always so clear-cut. Although many writers think you should
simply present and report your findings on the data you have
collected, others believe some evaluation and commentary on your data
may be appropriate and even necessary here. You and your teacher can
decide how strict you want to be in this decision.

There are specific conventions for creating tables, charts, and graphs
and organizing the information they contain. In general, you should
use these only when you are sure they will enlighten your readers
rather than confuse them. In the accompanying explanation and your
discussion, always refer to the graphic by number and explain
specifically what you are referring to. Give your graphic element a
descriptive caption as well. The rule of thumb for presenting a
graphic is first to introduce it by name, show it, and then interpret
it. You can consult a textbook, such as Lannon’s Technical Writing for
more information and guidance. The results section is usually written
in past tense.

\subsection{Quantitative\label{sec:quant}}
Quantitative information, data that can be measured, can
be presented systematically and economically in tables, charts, and
graphs. Quantitative information includes quantities and comparisons
of sets of data. If you are unfamiliar with the conventions, you may
find it challenging to present quantitative findings. You may include
some commentary to explain to your reader what your findings are and
how to read them.

\subsection{Qualitative\label{sec:qual}}
Qualitative information, which includes brief descriptions,
explanations, or instructions, can also be presented in prose
tables. This kind of descriptive or explanatory information, however,
is often presented in essay-like prose or even lists.

%/ =======================================================================================
\section{Discussion\label{sec:discuss}}
%/ =======================================================================================

Your discussion section should generalize on what you have learned
from your research. One way to generalize is to explain the
consequences or meaning of your results and then make your points that
support and refer back to the statements you made in your
introduction. Your discussion should be organized so that it relates
directly to your thesis. You want to avoid introducing new ideas here
or discussing tangential issues not directly related to the
exploration and discovery of your thesis. This section, along with the
introduction, is usually written in present tense.

%/ =======================================================================================
\subsection{Conclusions and Recommendations\label{sec:conclude}}
%/ =======================================================================================

Some academic research assignments might end with the discussion and
not need a separate conclusions and recommendations section. Often, in
shorter assignments, your conclusion is just a paragraph or two added
to the discussion section. In many of your research assignments,
however, you will be asked to provide your conclusions and
recommendations in your research paper.

Conclusions unify your research results and discussion and elaborate
on their significance to your thesis. Your conclusion ties your
research to your thesis, binding together all the main ideas in your
thinking and writing. By presenting the logical outcome of your
research and thinking, your conclusion answers your research inquiry
for you and your readers. Your conclusions should relate directly to
the ideas presented in your introduction section and not present any
new ideas.

You may be asked to present your recommendations separately in your
research assignment. If so, you will want to add some elements to your
conclusion section. For example, you may be asked to recommend a
course of action, make a prediction, propose a solution to a problem,
offer a judgment, or speculate on the implications and consequences of
your ideas. The conclusions and recommendations section is usually
written in present tense.

\nocite{savitsky:03}





%/ =======================================================================================
\section*{References\label{sec:cites}}
%/ =======================================================================================

\bibliography{soliday}

%/ =======================================================================================
%\section{Appendices\label{sec:apendix}}
%/ =======================================================================================


%  ==========================================================================================

%\newpage
%\include{bio} 

%\begin{verbatim}
%http://www2.fiu.edu/~sabar/enc1102/Research%20Paper%20Advice.htm
%http://tychousa8.umuc.edu/WRTG999A/index.html
%\end{verbatim}

\end{document}

%/ =======================================================================================
%/ **                          H E A D I N G S P E E D . T E X                          **
%  ======================================================================== END FILE =====
