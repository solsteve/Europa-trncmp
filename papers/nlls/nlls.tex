%/ ====================================================================== BEGIN FILE =====
%/ **                   N O N L I N E A R   L E A S T   S Q U A R E S                   **
%/ =======================================================================================
%/ **                                                                                   **
%/ **  Copyright (c) 2020, Stephen W. Soliday                                           **
%/ **                      stephen.soliday@trncmp.org                                   **
%/ **                      http://research.trncmp.org                                   **
%/ **                                                                                   **
%/ **  -------------------------------------------------------------------------------  **
%/ **                                                                                   **
%/ **  This program is free software: you can redistribute it and/or modify it under    **
%/ **  the terms of the GNU General Public License as published by the Free Software    **
%/ **  Foundation, either version 3 of the License, or (at your option)                 **
%/ **  any later version.                                                               **
%/ **                                                                                   **
%/ **  This program is distributed in the hope that it will be useful, but WITHOUT      **
%/ **  ANY WARRANTY; without even the implied warranty of MERCHANTABILITY or FITNESS    **
%/ **  FOR A PARTICULAR PURPOSE. See the GNU General Public License for more details.   **
%/ **                                                                                   **
%/ **  You should have received a copy of the GNU General Public License along with     **
%/ **  this program. If not, see <http://www.gnu.org/licenses/>.                        **
%/ **                                                                                   **
%/ ----- Modification History ------------------------------------------------------------
%/
%/  @file nlls.tex
%/   Provides
%/
%/  @author Stephen W. Soliday
%/  @date 2014-Jul-14
%/  @date 2020-Jun-03
%/
%/ =======================================================================================

\documentclass{article}

\usepackage{amsmath}
\usepackage{amsfonts}
\usepackage{amssymb}
\usepackage{graphicx}
\usepackage{wrapfig}
\usepackage{trncmp}
\usepackage{picinpar}
\usepackage{asiwp}
\usepackage{environ}
\usepackage{fancyvrb}

%  =======================================================================================
% Heading arguments are {volume}{year}{pages}{submitted}{published}{author-full-names}

\jmlrheading{Tutorial}{2014}{}{6/18}{-}{Stephen W.~Soliday}{TRNCMP/ARD/CCB}

% Short headings should be running head and authors last names

\ShortHeadings{Non-linear Least Squares}{Soliday}
\firstpageno{1}
%  =======================================================================================

\newcommand{\MAT}[1]{\GPP{#1}}
\newcommand{\VAR}[3]{#1^{(#2)}_{#3}}
\newcommand{\efrac}[2]{\displaystyle\frac{#1}{#2}}
\newcommand{\pfrac}[2]{\efrac{\partial#1}{\partial#2}}

\NewEnviron{tcequation}{%
  \begin{equation}\scalebox{1.4}
    {$\BODY$}
  \end{equation}
}


\begin{document}
%\bibliographystyle{plain}

\title{Non-linear Least Squares \\ with an example of: Transmitter Localization}
\author{ \name Stephen W.~Soliday \email research@trncmp.org %\\
%  \addr Tranquillitatis Computing \\
%  \addr Advanced Research Division\\
%  \addr Cognitive Computing Branch \\
}
\date{16-July-2014}

\maketitle

\begin{abstract}%   <- trailing '%' for backward compatibility of .sty file
  Non-linear Least Squares (NLLS) is a method of solving an over determined set of 
  non-linear simultaneous equations. 
  This paper derives the NLLS algorithm, beginning with the derivation for 
  Newton-Raphson method of root convergence for single dimensions. 
  A method for the pseudo inversion of a rectangular matrix is also presented. 
  Finally, a real world example of transmitter localization is presented.
\end{abstract}

\vspace{12pt}
\textbf{Keywords}: Taylor-series, Newton-Raphson, Jacobian, Least-squares, Non-linear, 
Regression, Over-determined, Range-only Localization

%/ =======================================================================================
\section{Introduction\label{sec:intro}}
%/ =======================================================================================

This paper provides a step by step explanation of root solving in 1 or more dimensions.
The first subsection will review the basic one-dimensional case, known as the 
Newton-Raphson method.
The next section will cover the extension to n-dimensions for a fully determined system.
A fully determined system has the same number of equations as unknown variables.
Finally a worked example for transmitter localization is presented.

%/ ---------------------------------------------------------------------------------------
\subsection{Background}
%/ ---------------------------------------------------------------------------------------

Presented here are three mathematical techniques that are required by these derivations.
This include a discussion on a Taylor series, Jacobian matrix and a  technique for the
 pseudo inversion of a rectangular matrix.

%/ ---------------------------------------------------------------------------------------
\subsubsection{Taylor Series\label{sec:taylor}}
%/ ---------------------------------------------------------------------------------------
Each of the following sections will begin by expressing the first few terms of the 
Taylor series in increasing dimensions.

The Taylor Series may be used to approximate a function about a given point.

\begin{tcequation}
  \FU{f}{x_0 + \epsilon} = 
  \FU{f}{x_0} + \FU{f'}{x_0}\epsilon + \frac{1}{2}\FU{f''}{x_0}\epsilon^2 + \hdots
\end{tcequation}


\begin{tcequation}
  \FU{f}{x_0 + \epsilon} = \sum^\infty_{n=0} \frac{1}{n!}\FU{f^{\GPP{n}}}{x_0}\epsilon^n
\end{tcequation}

For the rest of this paper only the first two terms will be used:

\begin{tcequation}
  \FU{f}{x_0 + \epsilon} \approx \FU{f}{x_0} + \FU{f'}{x_0}\epsilon
  \label{eq:taylor}
\end{tcequation}

%/ ---------------------------------------------------------------------------------------
\subsubsection{Jacobian Matrix\label{sec:jacobian}}
%/ ---------------------------------------------------------------------------------------

The Jacobian matrix generalizes the gradient of a scalar-valued function of multiple variables,
which itself generalizes the derivative of a scalar-valued function of a single variable. 
In other words, the Jacobian of a scalar-valued function of a single variable
$F:\RE \rightarrow \RE$ is just the first derivative.

\begin{tcequation}
  J = \frac{d}{d x}\FU{F}{x}
\end{tcequation}

And the Jacobian for a scalar-valued multivariable function
$F:\RE^n \rightarrow \RE$ is the gradient.

\begin{tcequation}
  J = \nabla F = 
  \frac{\partial}{\partial q_1}F +
  \frac{\partial}{\partial q_2}F + \hdots
  \frac{\partial}{\partial q_n}F
\end{tcequation}

Extending this to a system of scalar-valued multivariable functions, 
$F \equiv f_i:\RE^n \rightarrow \RE \ \ \forall i=1,2,\hdots m$
to express a Jacobian matrix.

\begin{tcequation}
  J = \left(
  \begin{array}{c}
    \frac{\partial}{\partial q_1} \\  \\ 
    \frac{\partial}{\partial q_2} \\
    \vdots \\
    \frac{\partial}{\partial q_n}
  \end{array}
  \right) \cdot \left( f_1, f_2, \hdots f_m \right)
\end{tcequation}

\begin{tcequation}
  J = 
  \left(
  \begin{array}{cccc}
    \frac{\partial}{\partial q_1} f_1 & \frac{\partial}{\partial q_2} f_1 &
    \hdots & \frac{\partial}{\partial q_n} f_1 \\  \\ 
    \frac{\partial}{\partial q_1} f_2 & \frac{\partial}{\partial q_2} f_2 &
    \hdots & \frac{\partial}{\partial q_n} f_2 \\ \\ 
    \vdots & \vdots & \ddots & \vdots \\ \\
    \frac{\partial}{\partial q_1} f_m & \frac{\partial}{\partial q_2} f_m &
    \hdots & \frac{\partial}{\partial q_n} f_m
  \end{array}
  \right)
\end{tcequation}

If $p$ is a point in $\RE^n$ and $F$ is differentiable at $p$, then its derivative at $p$ is 
given by $\FU{J_F}{p}$. In this case, the linear map described by $\FU{J_F}{p}$ is 
the best linear approximation of $F$ near the point $p$, in the sense that

\begin{tcequation}
  \FU{F}{x} = \FU{F}{p} + \FU{J_F}{p} \left(x-p\right) + \FU{o}{\left|x-p\right|}
\end{tcequation}

for $x$ close to $p$ and where $o$ is the little o-notation (for $x - p$) and
$\left|x-p\right|$ is the distance between $x$ and $p$.

Compare this to a Taylor series for a scalar function of a scalar argument, 
truncated to first order:

\begin{tcequation}
  \FU{f}{x} = \FU{f}{p} + \FU{f'}{p} \left( x - p \right) + \FU{o}{\left|x-p\right|}
\end{tcequation}

%/ ---------------------------------------------------------------------------------------
\subsubsection{Pseudo Inversion of a Rectangular Matrix\label{sec:rectangle}}
%/ ---------------------------------------------------------------------------------------

This section presents one method for performing the pseudo inversion of a rectangular 
matrix. Given the following linear mapping:

\begin{tcequation}
  \GPP{
    \begin{array}{cccc}
      Y_{1,1} & Y_{1,2} & \hdots & Y_{1,o} \\
      Y_{2,1} & Y_{2,2} & \hdots & Y_{2,o} \\
      \vdots & \vdots & \ddots & \vdots  \\
      Y_{e,1} & Y_{e,2} & \hdots & Y_{e,o} 
    \end{array}
  }_{e,o} = 
  \GPP{
    \begin{array}{cccc}
      X_{1,1} & X_{1,2} & \hdots & X_{1,i} \\
      X_{2,1} & X_{2,2} & \hdots & X_{2,i} \\
      \vdots & \vdots & \ddots & \vdots  \\
      X_{e,1} & X_{e,2} & \hdots & X_{e,i} 
    \end{array}
  }_{e,i} \cdot 
  \GPP{
    \begin{array}{cccc}
      A_{1,1} & A_{1,2} & \hdots & A_{1,o} \\
      A_{2,1} & A_{2,2} & \hdots & A_{2,o} \\
      \vdots & \vdots & \ddots & \vdots  \\
      A_{i,1} & A_{i,2} & \hdots & A_{i,o} 
    \end{array}
  }_{i,o}
\end{tcequation}

Compactly written as 

\begin{tcequation}
  Y_{e,o} = X_{e,i} \cdot A_{i,o}
  \label{eq:overlin}
\end{tcequation}

Where $e$ is the number of events.
$i$ is the number of input/independent variables in each event.
$o$ is the number of output/dependent variables in each event.

If $e = i$ then the system of linear equations is fully determined, and an exact 
solution may exist,
and the matrix $X$ is square and therefor directly invertable.

If $e < i$ then the system is under determined and can not be solved for a point 
solution but rather for hyper-planes.

If $e > i$ then the system is over determined and an approximate value may be ``fit''
to the data.
It will be come apparent in the final solution that this will resemble a version of least 
squares,

Start by multiplying both sides of equation~\ref{eq:overlin} with the transpose of $X$:

\begin{tcequation}
  \GPB{X_{i,e}^T \cdot Y_{e,o}}_{i,o} = \GPB{X_{i,e}^T \cdot X_{e,i}}_{i,i} \cdot A_{i,o}
\end{tcequation}

the subscripts indicate the row and column dimensions of the variables or groups,
making the matrix operations easier to follow.

Next multiply both sides by the inverse of the square $i \times i$ matrix
$\left[{X^T X}\right]$

\begin{tcequation}
  \GPB{X_{i,e}^T \cdot X_{e,i}}_{i,i}^{-1} \cdot \GPB{X_{i,e}^T \cdot Y_{e,o}}_{i,o} = 
  \GPB{X_{i,e}^T \cdot X_{e,i}}_{i,i}^{-1} \cdot \GPB{X_{i,e}^T \cdot X_{e,i}}_{i,i}
  \cdot A_{i,o}
\end{tcequation}

And simplify

\begin{tcequation}
  \GPB{X_{i,e}^T \cdot X_{e,i}}_{i,i}^{-1} \cdot \GPB{X_{i,e}^T \cdot Y_{e,o}}_{i,o} = A_{i,o}
\end{tcequation}

Without subscripts the final solution may be expressed as:

\begin{tcequation}
  A = \GPB{X^T X}^{-1} \cdot \GPB{X^T Y}
\label{eq:pseudoinv}
\end{tcequation}

%/ =======================================================================================
\section{Newton-Raphson (one dimension)\label{sec:newton}}
%/ =======================================================================================

The Newton-Raphson (N-R) method is used to find the roots of a non-linear equations.
This section will derive N-R for a single valued function of one independent variable:
$f:\RE\rightarrow\RE$

examine the function:

\begin{tcequation}
  y = \FU{f}{x}
\end{tcequation}

In order to solve this function using N-R an objective function is constructed:

\begin{tcequation}
  \FU{F}{x} = \FU{f}{x} - y
  \label{eq:obj1}
\end{tcequation}

The requirement for this algorithm is that: the first derivative of the 
function with respect to $x$ is known, and is continuous about zero.

\begin{figure}[h]
  \begin{center}
    \includegraphics*[width=1.75in]{figures/newton.eps}
    \caption{Follow a tangent to the X--axis\label{fig:newton}}
  \end{center}
\end{figure}

Figure~\ref{fig:newton} illustrates the basic approach. The first derivative is the 
slope of the curve at point $x$. That tangent may be followed to the x-axis and a new
$x$ is found. That value of $x$ is recursively applied until the difference between 
two successive estimations ( $x_t$ and $x_{t-1}$ ) falls below some user defined threshold.

Set up equation~\ref{eq:taylor} in terms of a single valued function of one independent
variable, using the objective function from equation~\ref{eq:obj1}.

\begin{tcequation}
  \FU{F}{x_0 + \epsilon} = \FU{F}{x_0} + \FU{F'}{x_0}\epsilon
\end{tcequation}

The goal is to minimize the objective function. In other words to find
$\FU{F}{x_{k+1}} = 0$

Solving for $\epsilon$:

\begin{tcequation}
  \epsilon = - \frac{F_k}{F'_k}
  \label{eq:inter1}
\end{tcequation}

Because this will be an iterative process, $\epsilon = x_{k+1} - x_k$
is substituted into equation~\ref{eq:inter1}:

\begin{tcequation}
  x_{k+1} = x_k - \frac{\FU{F}{x_k}}{\FU{F'}{x_k}}
  \label{eq:newton1}
\end{tcequation}

And in terms of the original function $y = \FU{f}{x}$

\begin{tcequation}
  x_{k+1} = x_k - \frac{1}{\FU{F'}{x_k}}\GPP{\FU{f}{x_k} - y}
\end{tcequation}

%/ ---------------------------------------------------------------------------------------
\subsection{ Example -- iterative algorithm for square root }
%/ ---------------------------------------------------------------------------------------

\begin{table}[h]
  \begin{center}
    \begin{tabular}{c|c|c}
      \textbf{\#} & \textbf{$x$} & \textbf{$x^2$} \\ \hline
      0 & 65.50000 & 4290.25000 \\
      1 & 33.75000 & 1139.06250 \\
      2 & 18.81574 & 354.03210 \\
      3 & 12.88900 & 166.12628 \\
      4 & 11.52635 & 132.85680 \\
      5 & 11.44581 & 131.00649 \\
      6 & 11.44552 & 131.00000 \\
      7 & 11.44552 & 131.00000
    \end{tabular}
  \end{center}
  \caption{Square Root of 131\label{tab:sqt131}}
\end{table}

Suppose you want to find the square root of an arbitrary real number.

\begin{tcequation}
  x = \sqrt{c}
\end{tcequation}

square both sides and place it in the form of equation~\ref{eq:obj1}

\begin{tcequation}
  0 = \FU{F}{x} = x^2 - c
\end{tcequation}

take the first derivative with respect to $x$:

\begin{tcequation}
  \frac{d F}{ d x} = 2 x
\end{tcequation}

Substitute this into equation~\ref{eq:newton1}

\begin{tcequation}
  x_{k+1} = x_k - \frac{x_k^2 - c}{2 x_k} 
\end{tcequation}

and simplify

\begin{tcequation}
  x_{k+1} = \frac{1}{2} \GPP{ x_k + \frac{c}{x_k}}
\end{tcequation}

Table~\ref{tab:sqt131} shows that it only takes 6 or 7 iterations to get reasonable
accuracy, from an initial guess of $x_0 = 131 / 2$.

%/ =======================================================================================
\section{Newton-Raphson (multi dimension - fully determined)\label{sec:newtonFD}}
%/ =======================================================================================

In this section an equation will be derived to apply the Newton-Raphson method to
multidimensional data. This will be the solution of a fully determined set of
simultaneous non-linear equations. By fully determined it is meant that the number of
equations is equal to the number of unknown variables.
Each of the $n$ equations will be a single valued function of $n$ independent variables:

\begin{tcequation}
  f_i:\RE^n\rightarrow\RE \ \ \forall i=1,2,\hdots,n
\end{tcequation}

\begin{tcequation}
  y_i=\FU{f_i}{\bar{x}} = \FU{f_i}{x_1,x_2,\hdots,x_n}
\end{tcequation}

In order to solve this system of equations using N-R, an objective function is defined:

\begin{tcequation}
  \FU{F_i}{\bar{x}} = y_i - \FU{f_i}{\bar{x}}
\end{tcequation}

Set up equation~\ref{eq:taylor} in terms of a system of $n$ simultaneous single valued
equations of $n$ independent variables. Substitute the first derivative in the Taylor
series with a Jacobian matrix, see Section~\ref{sec:jacobian}

\begin{tcequation}
  \FU{F}{\bar{x}_0+\epsilon} = \FU{F}{\bar{x}_0} + \FU{J_F}{\bar{x}_0}\cdot\epsilon
\end{tcequation}

again the goal is to solve for $\epsilon$ when $\FU{F}{\bar{x}_0+\epsilon} = 0$

\begin{tcequation}
  \FU{J_F}{\bar{x}_0}\cdot\epsilon = -\FU{F}{\bar{x}_0}
  \label{eq:temp2}
\end{tcequation}

The Jacobian in equation~\ref{eq:temp2} is square so just multiply both sides by its
inverse:

\begin{tcequation}
  \epsilon = -\FU{J_F}{\bar{x}_0}^{-1} \cdot \FU{F}{\bar{x}_0}
  \label{eq:inter2}
\end{tcequation}

Because this will be an iterative process, $\epsilon = x_{k+1} - x_k$ is substituted
into equation~\ref{eq:inter2}:

\begin{tcequation}
  \bar{x}_{k+1} = \bar{x}_k - \FU{J_F}{\bar{x}_k}^{-1} \cdot \FU{F}{\bar{x}_k}
\end{tcequation}

And in terms of the original function $y_i = \FU{f_i}{\bar{x}}$

\begin{tcequation}
  \bar{x}_{k+1} = 
  \bar{x}_k - \FU{J_F}{\bar{x}_k}^{-1} \cdot \GPP{\bar{y} - \FU{f}{\bar{x}_k}}
  \label{eq:newton2}
\end{tcequation}

Where $\bar{x}$ is an $1 \times n$ row vector, representing the desired solution of
unknown independent variables. 
$\bar{y}$ is an $n \times 1$ column vector, representing the constant measured dependent
variable.
$\FU{J_F}{\bar{x}_0}$ is an $n \times n$ Jacobian matrix representing the first 
derivative of the $n$ objective functions with respect to the $n$ unknown independent
variables.

A reasonable test for convergence would be the mean square error (MSE) of the difference
between $ \bar{x}_{k+1}$ and $\bar{x}_k$:

\begin{tcequation}
  \mbox{MSE} = \GPP{\bar{x}_{k+1} - \bar{x}_k}^T\cdot\GPP{\bar{x}_{k+1} - \bar{x}_k}
\end{tcequation}

%/ =======================================================================================
\subsection{Add a worked example here}
%/ =======================================================================================

Xxxxxxx xx xxxxx xx xxxxxxxxxxx xxxxx xxxxxx xxxx xxxxxxxx xxxxxxx.
Xxxxxxx xx xxxxx xx xxxxxxxxxxx xxxxx xxxxxx xxxx xxxxxxxx xxxxxxx.
Xxxxxxx xx xxxxx xx xxxxxxxxxxx xxxxx xxxxxx xxxx xxxxxxxx xxxxxxx.
Xxxxxxx xx xxxxx xx xxxxxxxxxxx xxxxx xxxxxx xxxx xxxxxxxx xxxxxxx.

Xxxxxxx xx xxxxx xx xxxxxxxxxxx xxxxx xxxxxx xxxx xxxxxxxx xxxxxxx.
Xxxxxxx xx xxxxx xx xxxxxxxxxxx xxxxx xxxxxx xxxx xxxxxxxx xxxxxxx.

%/ =======================================================================================
\section{Newton-Raphson (multi dimension - over determined)\label{sec:newtonOD}}
%/ =======================================================================================

In this section an equation will be derived to apply the Newton-Raphson (N-R) method to
multidimensional data. This will be the solution of an over determined set of
simultaneous non-linear equations. By over determined it is meant that the number of
equations is more (much more) than the number of unknown variables.
Each of the $n$ equations will be a single valued function of $m$ independent variables:

\begin{tcequation}
  f_i:\RE^m\rightarrow\RE \ \ \forall i=1,2,\hdots,n
\end{tcequation}

\begin{tcequation}
  y_i=\FU{f_i}{\bar{x}} = \FU{f_i}{x_1,x_2,\hdots,x_m}
\end{tcequation}

In order to solve this system of equations using N-R, an objective function is defined:

\begin{tcequation}
  \FU{F_i}{\bar{x}} = y_i - \FU{f_i}{\bar{x}}
\label{eq:odobj}
\end{tcequation}

Set up the Taylor Series, from Equation~\ref{eq:taylor}, in terms of a system of $n$ simultaneous single valued
equations of $m$ independent variables. 
Substitute the first derivative in the Taylor
Series with a Jacobian matrix, see Section~\ref{sec:jacobian}

\begin{tcequation}
  \FU{F}{\bar{x}_0+\epsilon} = \FU{F}{\bar{x}_0} + \FU{J_F}{\bar{x}_0}\cdot\epsilon
\end{tcequation}

again the goal is to solve for $\epsilon$ when $\FU{F}{\bar{x}_0+\epsilon} = 0$

\begin{tcequation}
  \FU{J_F}{\bar{x}_0}\cdot\epsilon = -\FU{F}{\bar{x}_0}
  \label{eq:temp2b}
\end{tcequation}

The Jacobian in equation~\ref{eq:temp2b} is not square, 
it is $n \times m$.
The normal matrix inversion is substituted with the pseudo inversion from Section~\ref{sec:rectangle}:

The following equation would be incorrect:

\begin{tcequation}
  \epsilon = -\FU{J_F}{\bar{x_0}}^{-1} \cdot \FU{F}{\bar{x_0}}
\end{tcequation}

So, it is expressed in the form of Equation~\ref{eq:pseudoinv}:

\begin{tcequation}
  \epsilon = -\GPB{\FU{J_F}{\bar{x_0}}^T \FU{J_F}{\bar{x_0}}}^{-1} \cdot \GPB{\FU{J_F}{\bar{x_0}}^T Y}
  \label{eq:inter2b}
\end{tcequation}

Because this will be an iterative process, $\epsilon = x_{k+1} - x_k$ is substituted
into equation~\ref{eq:inter2b}:

\begin{tcequation}
  \bar{x}_{k+1} = \bar{x}_k - \GPB{\FU{J_F}{\bar{x}_k}^T \FU{J_F}{\bar{x}_k}}^{-1} \cdot \GPB{\FU{J_F}{\bar{x}_k}^T Y}
\end{tcequation}

And in terms of the original function $y_i = \FU{f_i}{\bar{x}}$

\begin{tcequation}
  \bar{x}_{k+1} = 
  \bar{x}_k - \GPB{\FU{J_F}{\bar{x}_k}^T \FU{J_F}{\bar{x}_k}}^{-1} \cdot \GPB{\FU{J_F}{\bar{x}_k}^T \GPP{\FU{f}{\bar{x}_k} - \bar{y}}}
  \label{eq:newton2b}
\end{tcequation}

Where $\bar{x}$ is an $1 \times m$ row vector, representing the desired solution of
unknown independent variables. 
$\bar{y}$ is an $n \times 1$ column vector, representing the constant measurement
variable.
$\FU{J_F}{\bar{x}_0}$ is an $n \times m$ Jacobian matrix representing the first 
derivative of the $n$ objective functions with respect to the $m$ unknown independent
variables.

A reasonable test for convergence would be the mean square error (MSE) of the difference
between $ \bar{x}_{k+1}$ and $\bar{x}_k$:

\begin{tcequation}
  \mbox{MSE} = \GPP{\bar{x}_{k+1} - \bar{x}_k}^T\cdot\GPP{\bar{x}_{k+1} - \bar{x}_k}
\end{tcequation}

%/ =======================================================================================
\section{Transmitter Localization -- example of the over determined Non-linear Least Squares}
%/ =======================================================================================
\begin{figure}[ht]
\begin{center}
\includegraphics*[width=2.5in]{figures/LocalSetup.eps}
\caption{Setup for problem number 1.\label{fig:setup}}
\end{center}
\end{figure}

This example involves locating a transmitter using range only 
measurements\footnote{RTS-CTS timing can be used to determine range.
  Under ideal conditions RSSI may be calibrated for distance measurements.} from multiple 
receivers.
Figure~\ref{fig:setup} shows a transmitter T at its true location.
The points $Q_i$ are the locations of receivers where measurements are made. 
$r_i$ indicates the measured value of distance between the receiver and the true
location of the transmitter. 
\textbf{$\theta$} is the estimated location of the transmitter.
$d_i$ indicates the expected distance between the $i^{th}$  receiver and the estimated
Transmitter location.

The objective function consists of minimizing the sum of the square differences
between the measured distances and the estimated distances.

Let $\theta = \GPR{x,y}$ be the estimated location of the transmitter,
and $\theta_i = \GPR{x_i,y_i}$ be the location of the $i^{th}$ receiver.
The distance between this estimated location and the location of the $i^{th}$ receiver is:

\begin{tcequation}
\FU{d_i}{\theta} = \sqrt{\GPP{x_i - x}^2 + \GPP{y_i - y}^2}
\label{eq:euclid}
\end{tcequation}

From this definition and Equation~\ref{eq:odobj}, the objective function is:

\begin{tcequation}
 \FU{F_i}{\theta} = r_i - \FU{d_i}{\theta}
\label{eq:locobj}
\end{tcequation}

The iterative solution will come from Equation~\ref{eq:newton2}:

\begin{tcequation}
  \theta_{k+1} = 
  \theta_k - \GPB{\FU{J_F}{\theta_k}^T \FU{J_F}{\theta_k}}^{-1} \cdot 
  \GPB{\FU{J_F}{\theta_k}^T \GPP{r_i - \FU{d_i}{\theta_k}}}
\label{eq:oditer}
\end{tcequation}

The severance matrix for the estimation will be:

\begin{tcequation}
\FU{Cov}{\theta_k} = \sigma_k^2 \GPB{\FU{J_F}{\theta_k}^T \FU{J_F}{\theta_k}}^{-1}
\label{eq:cov}
\end{tcequation}

where $\sigma_k^2$ is estimated to be:

\begin{tcequation}
\sigma_k^2 = \frac{1}{n-3} \sum^n_{i=1} \GPB{r_i - \FU{d_i}{\theta_k}}^2
\end{tcequation}

The Jacobian is an $n \times 2$ matrix:

\begin{tcequation}
\FU{J_F}{\theta_k} = \MAT{\begin{array}{cc}
\frac{\partial\FU{F_1}{\theta_k}}{\partial x} & \frac{\partial\FU{F_1}{\theta_k}}{\partial y} \\ \\
\frac{\partial\FU{F_2}{\theta_k}}{\partial x} & \frac{\partial\FU{F_2}{\theta_k}}{\partial y} \\ \\
\vdots & \vdots \\
\frac{\partial\FU{F_n}{\theta_k}}{\partial x} & \frac{\partial\FU{F_n}{\theta_k}}{\partial y} \\ \\
\end{array}}_{n,2}
\label{eq:jacobA}
\end{tcequation}

\begin{tcequation}
\FU{J_F}{\theta_k}^T = \MAT{\begin{array}{cccc}
\frac{\partial\FU{F_1}{\theta_k}}{\partial x} & \frac{\partial\FU{F_2}{\theta_k}}{\partial x} & \cdots & \frac{\partial\FU{F_n}{\theta_k}}{\partial x} \\ \\
\frac{\partial\FU{F_1}{\theta_k}}{\partial y} & \frac{\partial\FU{F_2}{\theta_k}}{\partial y} & \cdots & \frac{\partial\FU{F_n}{\theta_k}}{\partial y} \\ \\
\end{array}}_{2,n}
\label{eq:jacobAT}
\end{tcequation}

From Equations~\ref{eq:euclid}~\&~\ref{eq:locobj}, the partial derivative of $\FU{F_i}{\theta_k}$ with respect to $x$ and $y$ are:

\begin{tcequation}
\frac{\partial\FU{F_i}{\theta_k}}{\partial x} = \frac{\GPP{x_i - x}}{\FU{d_i}{\theta_k}} \ \  \ \ \mbox{and} \ \  \ \ 
\frac{\partial\FU{F_i}{\theta_k}}{\partial y} = \frac{\GPP{y_i - y}}{\FU{d_i}{\theta_k}}
\end{tcequation}

Substituting these into Equations~\ref{eq:jacobA}~\&~\ref{eq:jacobAT}, and using the dot product, the transpose of the Jacobian doted with itself is:

\begin{tcequation}
\FU{J_F}{\theta_k}^T \cdot \FU{J_F}{\theta_k} = 
\MAT{\begin{array}{cc}
\sum^n_{i=1} \frac{\GPP{x_i - x}^2}{\FU{d_i}{\theta_k}^2} &
\sum^n_{i=1} \frac{\GPP{x_i - x}\GPP{y_i - y}}{\FU{d_i}{\theta_k}^2} \\ \\
\sum^n_{i=1} \frac{\GPP{x_i - x}\GPP{y_i - y}}{\FU{d_i}{\theta_k}^2} &
\sum^n_{i=1} \frac{\GPP{y_i - y}^2}{\FU{d_i}{\theta_k}^2} \\ \\
\end{array}}_{2,2}
\end{tcequation}

As a side note, the elements of the Jacobian are the angles $\alpha_i$ from the estimated position of the transmitter and the location of the receivers:

 \begin{tcequation}
\FU{J_F}{\theta_k}^T \cdot \FU{J_F}{\theta_k} = 
\MAT{\begin{array}{cc}
\sum^n_{i=1} \FU{\cos^2}{\alpha_i} &
\sum^n_{i=1} \FU{\cos}{\alpha_i}\FU{\sin}{\alpha_i} \\ \\
\sum^n_{i=1} \FU{\cos}{\alpha_i}\FU{\sin}{\alpha_i} &
\sum^n_{i=1} \FU{\sin^2}{\alpha_i}  \\ \\
\end{array}}_{2,2}
\end{tcequation}

 \centerline{\fbox{%
   \begin{minipage}{4.25 in}
     A $2 \times 2$ matrix with equal off diagonal elements has a trivial inverse:

     \begin{tcequation}
       \MAT{\begin{array}{cc} A & C \\ C & B \end{array}}^{-1} =
       \frac{1}{ A B - C^2}
       \MAT{\begin{array}{cc} B & -C \\ -C & A \end{array}}
     \end{tcequation}

     It is left as an exercise to the reader to combine these two equations:
 \end{minipage}}}

 \vspace{12 pt}

The right hand factor from Equation~\ref{eq:oditer} is:

\begin{tcequation}
\FU{J_F}{\theta_k}^T \cdot \GPP{r_i - \FU{d_i}{\theta_k}} =
\MAT{\begin{array}{c}
\sum^n_{i=1} \frac{\GPP{x_i - x}}{\FU{d_i}{\theta_k}}  \GPP{r_i - \FU{d_i}{\theta_k}}
\\ \\
\sum^n_{i=1} \frac{\GPP{y_i - y}}{\FU{d_i}{\theta_k}}  \GPP{r_i - \FU{d_i}{\theta_k}}
 \end{array}}_{2,1}
\end{tcequation}


{\bf WRITE OUT THE FINAL EQUATION HERE}





\section{Appendix}

%  ==========================================================================================
%
%  PROVIDED A DESCRIPTION OF  THE RELATIONSHIP BETWEEN ERROR ELLIPSES AND COVARIANCE MATRICES
%
%  ==========================================================================================

Provide a discussion of error ellipses


\nocite{Iosifescu80}

\bibliography{soliday}

%\begin{verbatim}
%http://www2.fiu.edu/~sabar/enc1102/Research%20Paper%20Advice.htm%
%
%http://tychousa8.umuc.edu/WRTG999A/index.html
%\end{verbatim}

%/ =======================================================================================

\newpage

\begin{wrapfigure}{l}{3cm}
\vspace{-14pt}
\includegraphics[width=3cm]{images/Soliday.eps}
\end{wrapfigure}

\textbf{Stephen W. Soliday, MSEE}--is a
research scientist with L3Harris Advanced Defense Solutions.
He is responsible for providing subject matter expertise for Machine
Learning, Advanced Algorithms, and Cyber Autonomy. He is a member of
L3’s newly formed AI Community of Practice.  In 2015, Mr. Soliday was
elected as an L3-NSS Engineering Fellow.  In 1992, Mr. Soliday earned
a Bachelor’s degree in physics and mathematics, concentrating on
computational astrophysics. In 1996, Mr. Soliday completed a Master of
Science in Electrical Engineering with a concentration on Machine
Leaning and Controls. His research involved autonomous behavior for
mobile robotics on a research grant from DARPA. Mr. Soliday has worked
for Texas Instruments, Raytheon, Titan, and L3 Technologies. In the
mid-1980’s Mr. Soliday served in the U.S. Army Air Defense.
Mr. Soliday has over 25 years of technical expertise in the
application of Machine Learning and Autonomy to: information
operations, weapon systems, unmanned vehicles, and C4ISR systems, for
both National and DOD programs. He has implemented and supported
technologies such as genetic algorithms, fuzzy-logic, neural-networks,
intelligent-agents, and Kalman-filters, to advanced research projects,
concept development programs, and production systems.  Mr. Soliday has
supported, designed, and led many computational modeling and
simulation projects. 


\end{document}

%/ =======================================================================================
%/ **                   N O N L I N E A R   L E A S T   S Q U A R E S                   **
%/ ======================================================================== END FILE =====
